\documentclass[a4paper,11pt]{kth-mag}
\usepackage[T1]{fontenc}
\usepackage{textcomp}
\usepackage{lmodern}
\usepackage[utf8]{inputenc}
\usepackage{csquotes}
\usepackage[swedish,english]{babel}
\usepackage{modifications}
\usepackage[backend=biber]{biblatex}
\bibliography{bibliography.bib}

\newenvironment{metatext}{%
  \textbf{$\hookrightarrow$}
  \begin{itshape}
}{
  \end{itshape}
  \newline
  \newline
  \useignorespacesandallpars
}

\def\useignorespacesandallpars#1\ignorespaces\fi{%
#1\fi\ignorespacesandallpars}

\makeatletter
\def\ignorespacesandallpars{%
  \@ifnextchar\par
    {\expandafter\ignorespacesandallpars\@gobble}%
    {}%
}
\makeatother

\title{Modular responsive web design}
\foreigntitle{Modulär responsiv webbutveckling}
\subtitle{Allowing responsive web modules to respond to custom criterias instead of only viewport size by implementing \emph{element queries}}
\author{Lucas Wiener \\ \lowercase{lwiener@kth.se}}
\date{February 2015}
\blurb{Master's Thesis at \textsc{csc}\\\hfill\\ Supervisors at \textsc{evry ab}: Tomas Ekholm \& Stefan Sennerö\\Supervisor at \textsc{csc}: Philipp Haller\\Examiner: Mads Dam}
\trita{TRITA xxx yyyy-nn}
\begin{document}
  \frontmatter
  \pagestyle{empty}
  \removepagenumbers
  \maketitle
  \selectlanguage{english}
  \begin{abstract}
    Abstract goes here.
  \end{abstract}
  \clearpage
  \begin{foreignabstract}{swedish}
    Sammanfattning ska vara här.
  \end{foreignabstract}
  \clearpage
  \tableofcontents*
  \mainmatter
  \pagestyle{newchap}
  \chapter{Introduction}
    \section{Targeted audience}
    \section{Problem statement}
    \section{Objective}
    \section{Methodology}
    \section{Delimitations}
    \section{Outline}
  \part{Background}

    \chapter{Browsers} 
      \begin{metatext}
        Browsers and the Internet is something that many people today take for granted.
        It is not longer the case that only computer scientists are browsing the web.
        Today the web is becoming increasingly important in both our personal and professional lives.
        This chapter will give a brief history of browsers and the rise of the web.
        It will also cover the role of browsers today, and what can be expected in the future.
        This section is a summary of \normalfont{\cite{internet_live_stats}\cite{internet_of_things}\cite{wiki_hypermedia}\cite{w3c_www}\cite{oed}}.
      \end{metatext}

      Before addressing the birth of the web, lets define the meaning of the concepets of the \emph{Internet}, \emph{Web} and \emph{World Wide Web}.
      The word internet can be translated to \emph{something between networks}. 
      When referring to the Internet (capitalized) it is usually the global decentralized internet used for communication between millions of networks using \textsc{tcp/ip}.
      Since the Internet is decentralized, there is no owner.
      Or in other words, the owners are all the network end-points which means all users of the Internet.
      One can argue that the owners of the Internet are the \textsc{isp}'s, providing the services and infrastructure making the Internet possible.
      On the other hand, the backbones of the Internet are usually co-founded by countries and companies.
      Or is it the \textsc{icann}\footnote{The Internet Corporation for Assigned Names and Numbers} organization which has the responsibility for managing the \textsc{ip} addresses in the Internet namespace?
      Clearly, the Internet wouldn't be what it is today without all the actors.
      The Internet lays the ground for many systems and applications, including the World Wide Web, file sharing and telephony.
      In 2014 the number of Internet users was measured to just below 3 billions, and estimations shows that we have surpassed 3 billions users today (no report for 2015 has been made yet).
      Users are here defined as humans having unrestricted acccess to the Internet.
      If one instead measures the number of connected entities (electronic devices that communicates through the Internet) the numbers are much higher.
      An estimation for 2015 of 25 billions connected entities has been made, and the estimation for 2020 is 50 billions.
      
      As already stated, the Word Wide Web (abbreviated \textsc{www} or \textsc{w3}) is a system that operates through the Internet.
      The World Wide Web is usually shortened to simply \emph{the web}.
      The web is a system for accessing interlinked hypertext documents and other media such as images and videos.
      Since not only hypertext is interlinked on the web, the term \emph{hypermedia} can be used as an extension to hypertext which also includes other nonlinear medium of information (images, videos, etc.).
      Although the term hypermedia has been around for a long time, the term hypertext is still being used as synonym for hypermedia.
      Further, the web can also be referred to as the universe of information accessible through the web system.
      Therefore, the web is both the system enabling sharing of hypermedia and also all of the accessible hypermedia itself.
      Hypertext documents are today more known by the name \emph{web pages} or simply \emph{pages}.
      Multiple related pages compose a \emph{web site} or simply a \emph{site} and are usually hosted from the same domain.
      To transfer the resources between computers the protocol \textsc{http} is used.
      Typically the way of retrieving resources on the web is by using a \emph{web browser} or simply a \emph{browser}.
      Browsers handles the fetching, parsing and rendering of the hypertext (more about this in section~\ref{sec:browsers}).
      
      \section{The origin of the web}
        \begin{metatext}
          Since the web is a system operating on top of the Internet, it is needed to first investigate the origin of the Internet.
          This can be viewed from many angles and different aspects need to be taken into consideration.
          With that in mind, the origin of the Internet is not something easily pinned down and what will be presented here will be more of a technically interesting history.
          This section is a summary of \normalfont{\cite{overview_of_tcp_ip}\cite{internetsociety_history_internet}\cite{internet_maps}\cite{historyofthings_internet}}.
        \end{metatext}

        In the early 1960's \emph{packet switching} was being researched, which is a prerequisite of internetworking.
        With packet switching in place, the very important ancestor of the Internet \textsc{arpanet} (Advanced Research Projects Agency Network) was developed, which was the first network to implement the \textsc{tcp/ip} protocol suite.
        The \textsc{tcp/ip} protocol suite together with packet switching are fundamental technologies of the Internet.
        \textsc{arpanet} was funded by the United States Department of Defense (DoD) in order to interconnect their research sites in the United States.
        The first nodes of \textsc{arpanet} was installed at four major universities in the western United States in 1969 and two years later the network spanned the whole country.
        The first public demonstration of \textsc{arpanet} was held at the International Computer Communication Conference (ICCC) in 1972.
        It was also at this time the email system was introduced, which became the largest network application for over a decade.
        In 1973 the network had international connections to Norway and London via a sattelite link.
        At this time information was exchanged with the File Transfer Protocol (\textsc{ftp}), which is a protocol to transfer files between hosts.
        This can be viewed as the first generation of the Internet. With around 40 nodes, operating with raw file transfers between the hosts it was mostly used by the academic community of the United States.

        The number of nodes and hosts of \textsc{arpanet} increased slowly, mainly due to the fact that it was a centralized network owned and operated by the \textsc{us} military.
        In 1974 the \textsc{tcp/ip} stack was proposed in order to have a more robust and scalable system for end-to-end network communication.
        The \textsc{tcp/ip} stack is a key technology for the decentralization of the \textsc{arpanet}, to allow the massive expandation of the network that later happened.
        In 1983 \textsc{arpanet} switched to the \textsc{tcp/ip} protocols, and the network was split in two.
        One network was still called \textsc{arpanet} and was to be used for research and development sites.
        The other network was called \textsc{milnet} and was used for military purposes.
        The decentralization event was a key point and perhaps the birth of the Internet.
        The Computer Science Network (\textsc{csnet}) was funded by the National Science Foundation (\textsc{nsf}) in 1981 to allow networking benefits to academic insitutions that could not directly connect to \textsc{arpanet}.
        After the event of decentralizing \textsc{arpanet}, the two networks were connected among many other networks.
        In 1985 \textsc{nsf} started the National Science Foundation Network (nsfnet) program to promote advanced research and education networking in the \textsc{us}.
        To link the supercomputing centers funded by \textsc{nsf} the \textsc{nsfnet} serverd as a high speed and long distance backbone network.
        As more networks and sites were linked by the \textsc{nsfnet} network, it became the first backbone of the Internet.
        In 1992, around 6000 networks were connected to the \textsc{nsfnet} backbone with many international networks.
        To this point, the Internet was still a network for scientists, academic institutions and technology enthusiasts.
        Mainly, because \textsc{nsf} had stated that \textsc{nsfnet} was a network for non-commercial traffic only.
        In 1993 \textsc{nsf} decided to go back to funding research in supercomputing and high-speed communcations instead of funding and running the Internet backbone.
        That, along with an increasing preassure of commercializing the Internet let to another key event in the history of the Internet - the privatization of the \textsc{nsfnet} backbone.

        In 1994, the \textsc{nsfnet} was systematically privatized while making sure that no actor owned too much of the backbone in order to create constructive market competition.
        With the Internet decentralized and privatized regular people started using it as well as companies.
        Backbones were built across the globe, more international actors and organizations appeared and eventually the Internet as we know it today came to exist.

        \subsection{The World Wide Web}
          \begin{metatext}
            Now that the history of the Internet has been described, it is time to talk about the birth of the World Wide Web.
            Here the initial ideas of the web will be described and how it became a global standard.
            This subsection is a summary of \normalfont{\cite{wiki_www}\cite{wiki_gopher}\cite{webdevnotes_history_of_the_internet}\cite{webdevnotes_www_basics}\cite{historyofthings_internet}}.
          \end{metatext}

          Recall that the way of exchanging information was to upload and download files between clients and hosts with \textsc{ftp}.
          If a document downloaded was referring to another document, the user had to manually find the server that hosted the other document and download it manually.
          This was a poor way of digesting information and documents that linked to other resources.
          In 1989 a proposal for a communication system that allowed interlinked documents was submitted to the management at \textsc{cern}.
          The idea was to allow links embedded in text documents, to enable users to view the linked document by clicking it.
          A quote from the draft:
          \begin{quote}
            Imagine, then, the references in this document all being associated with the network address of the thing to which they referred, so that while reading this document you could skip to them with a click of the mouse.
          \end{quote}
          This catches the whole essence of the web in a sentence --- to interlink resources in an user friendly way.
          The proposal describes that such text embedded links would be hypertext.
          It continues to explain that interlinked resources does not need to be limited to text documents, multimedia such as images and videos can also be interlinked which would similarly be hypermedia.
          The concept of browsers is described, with a client-server model the browser would fetch the hypertext documents, parse them and handle the fetching of all media linked in the hypertext.

          In 1990, the Hypertext Transfer Protocol (\textsc{http}), the Hypertext Markup Language (\textsc{html}), a browser and a web server had been created and the web was born.
          One year later the web was introduced to the public and in 1993 over five hundred international web servers existed.
          It was stated in 1994 that the web was to be free without any patents or royalties.
          At this time the Wolrd Wide Web Consortium (\textsc{w3c}) was founded by Berners-Lee with support from the Defense Advanced Research Projects Agency and the European Commission.
          The organization comprised of companies and individuals that wanted to standardize and improve the web.

      \section{The need for browsers}
      \label{sec:browsers}

    \chapter{Web development}
      \section{From documents to applications}
      \section{Responsive web design}
      \section{Modularity}
    \chapter{Modular development}
      \section{Web Components}
  \part{Theory}
    \chapter{Rendering engines}
    \chapter{Element queries}
  \part{Third-party framework}
    \chapter{Analysis of approaches}
    \section{Current implementations}
    \chapter{API design}
    \chapter{Implementation}
  \part{Result}
    \chapter{Discussion}
  \printbibliography
  \appendix
  \addappheadtotoc
  \chapter{RDF}\label{appA}
    \begin{figure}[ht]
      \begin{center}
        And here is a figure
        \caption{\small{Several statements describing the same resource.}}\label{RDF_4}
      \end{center}
    \end{figure}
  that we refer to here: \ref{RDF_4}
\end{document}
