\documentclass[a4paper,11pt]{kth-mag}
\usepackage[T1]{fontenc}
\usepackage{textcomp}
\usepackage{lmodern}
\usepackage[utf8]{inputenc}
\usepackage{csquotes}
\usepackage[swedish,english]{babel}
\usepackage{modifications}
\usepackage[backend=biber]{biblatex}
\bibliography{bibliography.bib}

\newenvironment{metatext}{%
  \textbf{$\hookrightarrow$}
  \begin{itshape}
}{
  \end{itshape}
  \newline
  \newline
  \useignorespacesandallpars
}

\def\useignorespacesandallpars#1\ignorespaces\fi{%
#1\fi\ignorespacesandallpars}

\makeatletter
\def\ignorespacesandallpars{%
  \@ifnextchar\par
    {\expandafter\ignorespacesandallpars\@gobble}%
    {}%
}
\makeatother

\title{Modular responsive web design}
\foreigntitle{Modulär responsiv webbutveckling}
\subtitle{Allowing responsive web modules to respond to custom criterias instead of only viewport size by implementing \emph{element queries}}
\author{Lucas Wiener \\ \lowercase{lwiener@kth.se}}
\date{February 2015}
\blurb{Master's Thesis at \textsc{csc}\\\hfill\\ Supervisors at \textsc{evry ab}: Tomas Ekholm \& Stefan Sennerö\\Supervisor at \textsc{csc}: Philipp Haller\\Examiner: Mads Dam}
\trita{TRITA xxx yyyy-nn}
\begin{document}
  \frontmatter
  \pagestyle{empty}
  \removepagenumbers
  \maketitle
  \selectlanguage{english}
  \begin{abstract}
    Abstract goes here.
  \end{abstract}
  \clearpage
  \begin{foreignabstract}{swedish}
    Sammanfattning ska vara här.
  \end{foreignabstract}
  \clearpage
  \tableofcontents*
  \mainmatter
  \pagestyle{newchap}
  \chapter{Introduction}
    \section{Targeted audience}
    \section{Problem statement}
    \section{Objective}
    \section{Methodology}
    \section{Delimitations}
    \section{Outline}
  \part{Background}

    \chapter{Browsers} 
      \begin{metatext}
        Browsers and the Internet is something that many people today take for granted.
        It is not longer the case that only computer scientists are browsing the web.
        Today the web is becoming increasingly important in both our personal and professional lives.
        This chapter will give a brief history of browsers and the rise of the web.
        It will also cover the role of browsers today, and what can be expected in the future.
      \end{metatext}

      Before addressing the birth of the web, lets define the meaning of the concepets of the \emph{Internet}, \emph{Web} and \emph{World Wide Web}.
      The word internet can be translated to \emph{something between networks}. 
      When referring to the Internet (capitalized) it is usually the global decentralized internet used for communication between millions of networks using \textsc{tcp/ip}.
      Since the Internet is decentralized, there is no owner.
      Or in other words, the owners are all the network end-points which means all users of the Internet.
      One can argue that the owners of the Internet are the \textsc{isp}'s, providing the services and infrastructure making the Internet possible.
      On the other hand, the backbones of the Internet are usually co-founded by countries and companies.
      Or is it the \textsc{icann}\footnote{The Internet Corporation for Assigned Names and Numbers} organization which has the responsibility for managing the \textsc{ip} addresses in the Internet namespace?
      Clearly, the Internet wouldn't be what it is today without all the actors.
      The Internet lays the ground for many systems and applications, including the World Wide Web, file sharing and telephony.
      In 2014 the number of Internet users was measured to just below 3 billions, and estimations shows that we have surpassed 3 billions users today (no report for 2015 has been made yet) \cite{internet_live_stats}.
      Users are here defined as humans having unrestricted acccess to the Internet \cite{internet_live_stats}.
      If one instead measures the number of connected entities (electronic devices that communicates through the Internet) the numbers are much higher.
      An estimation for 2015 of 25 billions connected entities has been made, and the estimation for 2020 is 50 billions \cite{internet_of_things}.
      
      As already stated, the Word Wide Web (abbreviated \textsc{www} or \textsc{w3}) is a system that operates through the Internet.
      The World Wide Web is usually shortened to simply \emph{the web}.
      The web is a system for accessing interlinked hypertext documents and other media such as images and videos.
      Since not only hypertext is interlinked on the web, the term \emph{hypermedia} can be used as an extension to hypertext which also includes other nonlinear medium of information (images, videos, etc.) \cite{wiki_hypermedia}.
      Although the term hypermedia has been around for a long time, the term hypertext is still being used as synonym for hypermedia.
      Further, the web can also be referred to as the universe of information accessible through the web system.
      Therefore, the web is both the system enabling sharing of hypermedia and also all of the accessible hypermedia itself \cite{w3c_www}.
      Hypertext documents are today more known by the name \emph{web pages} or simply \emph{pages} \cite{oed}.
      Multiple related pages compose a \emph{web site} or simply a \emph{site} and are usually hosted from the same domain \cite{oed}.
      To transfer the resources between computers the protocol \textsc{http} is used.
      Typically the way of retrieving resources on the web is by using a \emph{web browser} or simply a \emph{browser}.
      Browsers handles the fetching, parsing and rendering of the hypertext (more about this in section~\ref{sec:browsers}).
      
      \section{The origin of the web}
        \begin{metatext}
          Since the web is a system operating on top of the Internet, it is needed to first investigate the origin of the Internet.
          This can be viewed from many angles and different aspects need to be taken into consideration.
          With that in mind, the origin of the Internet is not something easily pinned down and what will be presented here will be more of a technically interesting history.
        \end{metatext}

        In the early 1960's \emph{packet switching} was being researched, which is a fundamental networking technology behing the Internet \cite{overview_of_tcp_ip}.
        With packet switching in place, the very important ancestor of the Internet \textsc{arpanet} (Advanced Research Projects Agency Network) was developed, which was the first network to implement the \textsc{tcp/ip} protocol suite \cite{overview_of_tcp_ip}.
        The \textsc{tcp/ip} protocol suite together with packet switching are fundamental technologies of the Internet.
        \textsc{arpanet} was funded by the United States Department of Defense in order (DoD) to interconnect their research sites in the United States \cite{overview_of_tcp_ip}.
        The first nodes of \textsc{arpanet} was installed at four major universities in the western United States in 1969 \cite{overview_of_tcp_ip}.
        In 1971 the network spanned the whole country and had in 1973 connections to Europe \cite{overview_of_tcp_ip}.
        The first public demonstration of \textsc{arpanet} was held at the International Computer Communication Conference (ICCC) in 1972 \cite{internetsociety_history_internet}.
        It was also at this time the email system was introduced, which became the largest network application for over a decade \cite{internetsociety_history_internet}.

      \section{The need for browsers}
      \label{sec:browsers}

    \chapter{Web development}
      \section{From documents to applications}
      \section{Responsive web design}
      \section{Modularity}
    \chapter{Modular development}
      \section{Web Components}
  \part{Theory}
    \chapter{Rendering engines}
    \chapter{Element queries}
  \part{Third-party framework}
    \chapter{Analysis of approaches}
    \section{Current implementations}
    \chapter{API design}
    \chapter{Implementation}
  \part{Result}
    \chapter{Discussion}
  \printbibliography
  \appendix
  \addappheadtotoc
  \chapter{RDF}\label{appA}
    \begin{figure}[ht]
      \begin{center}
        And here is a figure
        \caption{\small{Several statements describing the same resource.}}\label{RDF_4}
      \end{center}
    \end{figure}
  that we refer to here: \ref{RDF_4}
\end{document}
